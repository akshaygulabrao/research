\section{Path-Planning}
The most essential function for an autonomous drone is path-planning. It must be able to move through space efficiently without colliding with any obstacles. There are generally 3 distinct problems within this framework: (1) Path Navigation, (2) Motion Planning, and (3) Surveillance.

Path navigation involves navigating from a source to a destination through obstacles. The goal is a sequence of waypoints that minimize the distance traveled to go from the source to the destination. Then a robot can go in a straight line to each waypoint, and when reached, travels in a straight line to the next waypoint until it reaches the destination.

Motion planning is similar to path navigation because the goal is still to navigate from a source position to a destination position, but differs from path navigation because it requires motor instructions for each timestep. Each motor requires unique instructions. The precise controls make the problem extremely difficult in space-constrained environments, where there is very little free space. Another difficult scenario is time-constrained environments, where the most critical success factor is time to target.

Surveillance is similar to path navigation in that the instructions involve a list of waypoints that the robot navigates towards in a straight line, but different in that there is no given starting point or ending point. The idea is that the drone should independently patrol a mission area, visiting important areas frequently while occasionally visiting other areas.

\subsection{Path Navigation}
The foundational work in path navigation is the A* (A-star) search algorithm, which involves intelligently exploring locations to find a route from the source to the destination. \cite{hart1968formal} The A* search algorithm always returns an optimal solution given an admissible heuristic function. A popular heuristic is distance, because it never overestimates the distance to the goal. However, the problem with the A* search is that when used as a path-planning algorithm, it doesn't quantize each grid cell intelligently, leading to extremely expensive computations for simple navigation areas. Figure 1 displays the internal representation of A* search.
\begin{figure}[h]
    \centering
    \includegraphics[width=0.5\textwidth]{images/best-first-search.png}
    \caption{An internal representation of Euclidean space in the A* search algorithm. Source: \url{http://theory.stanford.edu/~amitp/GameProgramming/AStarComparison.html}}
    \label{fig:astar}
\end{figure}

The inefficiency comes from the fact that each grid cell is naively the same area even when it is known prior that there is no obstacle between them. This leads to A* spending lots of time in obvious segments. Originally, I was responsible for implementing this algorithm for drone navigation in 2D space. The idea behind the mission was to maintain a coherent world view with multiple patrolling drones while staying outside of known danger zones. With my prior knowledge in computational geometry and path-planning, I introduced my manager to visibility graphs, and provided a mature implementation for the drone to use, introducing a significant algorithmic advantage over the A* search algorithm. It is hard to quantify how large the difference between the performance of A* and visibility graphs is. It allowed for a completely different representation. It turned 2D Euclidean space to a small graph with all the visibility relationships implicitly encoded. Complex areas that took minutes to find the best path now finished instantly.

Visibility graphs preprocess the Euclidean space given a list of obstacles. \cite{lozano1979algorithm} Each obstacle is represented by a list of points with the last point being identical to the first point. Visibility graphs convert the Euclidean space pathfinding problem to a graph-based pathfinding problem because of the axiom that the shortest path between two points is a straight line. If there is an obstacle in between the two points, then the shortest path will touch an obstacle. This axiom has a non-obvious conclusion: the shortest path can be found by traversing between the source, obstacle vertices, and the destination.

The naive algorithm is the simplest approach. It runs in $\mathcal{O}(n^4)$ time. Here, $n$ represents the number of points: all obstacle vertices, the source, and the destination. It works by checking if traversing straight from any one vertex to another is a valid operation. The complexity comes from the fact that each vertex pair must be checked for visibility by iterating through every other vertex pair to check for any intersections. It takes $n^2$ iterations to check each vertex pair, and $n^2$ vertex pairs to check. It is considered valid if there are no obstacle vertices that cross between the points being checked. Figure~\ref{fig:code} shows the pseudocode of how the visibility graph works. The most difficult part happens on line 10, when verifying whether any obstacle intersects the line. After the visibility graph is generated, it constructs a more efficient pathfinding represented shown in Figure~\ref{fig:visual}. It is impossible for the middle of an obstacle edge to be part of the shortest path between two points.

\begin{figure}[h]
    \centering
    \includegraphics[width=0.5\textwidth]{images/naive.png}
    \caption{Rough pseudocode to the high level idea of visibility graphs}
    \label{fig:code}
\end{figure}



\begin{figure}[h]
    \centering
    \includegraphics[width=0.5\textwidth]{images/visual.png}
    \caption{Representation of visibility graph}
    \label{fig:visual}
\end{figure}

Chan's github repository implemented the basic idea of visibility graphs which was very useful to me. \cite{chan2018vgraph} However, the code in the repository had a critical flaw. It automatically assumed that the shortest path would never involve staying on the polygon for two adjacent segments. There were no edges checked between points on the same polygon, only points going from one polygon to another. Any time there was only one polygon between the source and the destination, the algorithm would fail. I modified this work to add that adjacent points on the same polygon are visible.
